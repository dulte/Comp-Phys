\documentclass[a4paper, 10pt]{article}
\usepackage[utf8]{inputenc}
\usepackage{verbatim}
\usepackage{listings}
\usepackage{graphicx}
\usepackage{a4wide}
\usepackage{color}
\usepackage{amsmath}
\usepackage{amssymb}
\usepackage[dvips]{epsfig}
\usepackage[toc,page]{appendix}
\usepackage[T1]{fontenc}
\usepackage{cite} % [2,3,4] --> [2--4]
\usepackage{shadow}
\usepackage{hyperref}
\usepackage{titling}
\usepackage{marvosym }
\usepackage{subcaption}
\usepackage[noabbrev]{cleveref}

\renewcommand{\topfraction}{.85}
\renewcommand{\bottomfraction}{.7}
\renewcommand{\textfraction}{.15}
\renewcommand{\floatpagefraction}{.66}
\renewcommand{\dbltopfraction}{.66}
\renewcommand{\dblfloatpagefraction}{.66}
\setcounter{topnumber}{9}
\setcounter{bottomnumber}{9}
\setcounter{totalnumber}{20}
\setcounter{dbltopnumber}{9}


\setlength{\droptitle}{-10em}   % This is your set screw

\setcounter{tocdepth}{2}

\lstset{language=c++}
\lstset{alsolanguage=[90]Fortran}
\lstset{basicstyle=\small}
\lstset{backgroundcolor=\color{white}}
\lstset{frame=single}
\lstset{stringstyle=\ttfamily}
\lstset{keywordstyle=\color{red}\bfseries}
\lstset{commentstyle=\itshape\color{blue}}
\lstset{showspaces=false}
\lstset{showstringspaces=false}
\lstset{showtabs=false}
\lstset{breaklines}
\title{FYS3150 - Project 5\\
Financial Modelling}
\author{Daniel Heinesen, Gunnar Lange}
\begin{document}
\maketitle
\begin{abstract}
HEI
\end{abstract}
\tableofcontents
\section{Introduction}
\section{Theoretical models}
In this section we introduce the theoretical model necessary to model financial agents interacting among each other.
\subsection{Modelling the interaction between two financial agents}
We will begin with a simple model for the interaction of two agents. Assume that we have picked, at random, two agents $i$ and $j$, each with wealth given by $m_i$ and $m_j$. We will then model the interaction between this pair by drawing a random number, $\epsilon$ $\in [0,1]$, from a uniform distribution,  and redistributing the wealth among the two according to the equations:
\begin{equation}
m_1'=\epsilon(m_1+m_2)
\end{equation}
\begin{equation}
m_2'=(1-\epsilon)(m_1+m_2)
\end{equation}
Notice that the total wealth is conserved in this interaction, seeing as:
$$m_1'+m_2'=\epsilon(m_1+m_2)+(1-\epsilon)(m_1+m_2)=m_1+m_2$$
Thus we are only redistributing the wealth in the economy.\\
\subsection{First modification: implementing savings in the economy}
We expand our model by including the possibility of our agents retaining a certain amount of money at every transaction. This is modelled by a parameter $\beta$, which describes the fraction of money saved by each agent in the transaction. Our models can then be written as:
\begin{equation}
\begin{split}
m_i'=\lambda m_i+\epsilon(1-\lambda)(m_i+m_j) \\
m_j'=\lambda m_j+(1-\epsilon)(1-\lambda)(m_i+m_j)
\end{split}
\end{equation}
Note that the total money is still conserved, seeing as:
$$m_i'+m_j'=\lambda m_i+\epsilon(1-\lambda)(m_i+m_j)+\lambda m_j+(1-\epsilon)(1-\lambda)(m_i+m_j)=\lambda (m_i+m_j)+(1-\lambda)(m_i+m_j)=m_i+m_j$$
\subsection{Second modification: Including the effect of proximity}
Our second modification is inspired by an idea published \textbf{HERE}. We introduce a bias on the interaction of the agents, by making it more likely for agents with comparable amount of wealth to interact. 
\subsection{Analytic solutions}

\section{Methods}
\subsection{Implementing the Monte Carlo simulation}
To ensure that we get good results, we run multiple simulations, and average the final distribution in each of these simulations. This is summarized in Pseudo-code below:
\lstinputlisting{Pseudo_code_simple_walkers.cpp}
\section{Results}
\section{Discussion}
\section{Conclusion}
\end{document}

