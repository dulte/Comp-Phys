\documentclass[a4paper,10pt,english]{article}
\usepackage[utf8]{inputenc}
\usepackage[english]{babel}
\usepackage{gensymb}
\usepackage{float}
\usepackage{array}
\DeclareUnicodeCharacter{00A0}{ }

\usepackage{amsmath,graphicx,varioref,verbatim,amsfonts,geometry}

\usepackage[usenames,dvipsnames,svgnames,table]{xcolor}

\usepackage[colorlinks]{hyperref}

\usepackage{bm}
\newcolumntype{L}{>{$}l<{$}} % math-mode version of "l" column type  
\newcolumntype{C}{>{$}c<{$}} % math-mode version of "c" column type  
\newcolumntype{R}{>{$}r<{$}} % math-mode version of "r" column type  

\newcommand{\uvec}[1]{\hat{#1}}

\newcommand{\pd}[1]{\frac{\partial}{\partial #1}}
\newcommand{\pdt}[2]{\frac{\partial #1}{\partial #2}}
\newcommand{\pddt}[2]{\frac{\partial^2 #1}{\partial #2^2}}
\newcommand{\tpddt}[2]{\tfrac{\partial^2 #1}{\partial #2^2}}
\newcommand{\rar}{\rightarrow}
\newcommand{\Rar}{\Rightarrow}
\newcommand{\curl}[1]{\nabla \times \vec{#1}}
\newcommand{\divr}[1]{\nabla \cdot \vec{#1}}
\newcommand{\lapl}{\nabla^2}
\newcommand{\myequals}[1]{\underline{\underline{#1}}}


\newcommand{\crossprod}[6]
{   
    \begin{vmatrix}  
        \uvec{i}    &\uvec{j}   &\uvec{k}  \\
        {#1}          &{#2}         &#3        \\
        {#4}          &{#5}         &#6        \\
    \end{vmatrix}
}
\newcommand{\curlmat}[3]
{
    \begin{vmatrix}  
        \uvec{i}    &\uvec{j}   &\uvec{k}  \\
        \pd{x}      &\pd{y}     &\pd{z}    \\
        #1          &{#2}         &{#3}        \\
    \end{vmatrix}
}



\setlength{\parindent}{4mm}
%\setlength{\parskip}{1.5mm}

%Color scheme for listings
\usepackage{textcomp}
\definecolor{listinggray}{gray}{0.9}
\definecolor{lbcolor}{rgb}{0.9,0.9,0.9}

%Listings configuration
\usepackage{listings}

\lstset{
	backgroundcolor=\color{lbcolor},
	tabsize=4,
	rulecolor=,
	language=python,
        basicstyle=\scriptsize,
        upquote=true,
        aboveskip={1.5\baselineskip},
        columns=fixed,
	numbers=left,
        showstringspaces=false,
        extendedchars=true,
        breaklines=true,
        prebreak = \raisebox{0ex}[0ex][0ex]{\ensuremath{\hookleftarrow}},
        frame=single,
        showtabs=false,
        showspaces=false,
        showstringspaces=false,
        identifierstyle=\ttfamily,
        keywordstyle=\color[rgb]{0,0,1},
        commentstyle=\color[rgb]{0.133,0.545,0.133},
        stringstyle=\color[rgb]{0.627,0.126,0.941}
}
% Custom commands        

\newcommand{\RNum}[1]{\uppercase\expandafter{\romannumeral #1\relax}}
\newcommand{\pagesplit}[2]{\noindent
    \begin{minipage}{0.5\linewidth}
        #1 
    \end{minipage}%
    \begin{minipage}{0.5\linewidth}
        #2
    \end{minipage}\par\vspace{\belowdisplayskip}
}


\newcounter{subproject}
\renewcommand{\thesubproject}{\alph{subproject}}
\newenvironment{subproj}{
\begin{description}
\item[\refstepcounter{subproject}(\thesubproject)]
}{\end{description}}

%\renewcommand{\labelenumi}{\alph{enumi}}
\renewcommand{\thesubsection}{\alph{subsection}}
\graphicspath{{~/uio/fysmek1100/oblig3/bilder}}

\title{FYS3150 - Project 1} 
\author{Halvard Sutterud} 
\begin{document}

\maketitle
\begin{abstract}
    In this project we solved the one-dimensional Poisson equation with
    Dirichlet boundary conditions. 
\end{abstract}

%\tableofcontents
\newpage

We will approximate the second order derivative in the general case of the one-dimensional
Poisson equation with $v_i$, given $n$ grid points $x_i = ih$ where
$h=1/(n+1)$. The approximation is

\[ 
    - \frac{v_{i+1} + v_{i-1} - 2v_i}{h^2} = f_i \hspace{0.5cm}
    \mathrm{for} \hspace{0.1cm}i = 1, ... , n.
\]
        
Where $f_i$ = $f(x_i)$.Multiplying by $h^2$, we then have $n$ linear
equations, which can be written as a matrix equation on the form $A\vec{v}
= \vec{\tilde b}$ where $\bold{A}$ is an $n\times n$ tridiagonal matrix and
$\tilde b_i = h^2 f$.  The factors $a_ij$ in the matrix corresponds to the
factor of $v_i$ in the equation number $j$, and the matrix is rewritten as

\begin{equation}
    {\bf A} = \left(\begin{array}{cccccc}
                           2& -1& 0 &\dots   & \dots &0 \\
                           -1 & 2 & -1 &0 &\dots &\dots \\
                           0&-1 &2 & -1 & 0 & \dots \\
                           & \dots   & \dots &\dots   &\dots & \dots \\
                           0&\dots   &  &-1 &2& -1 \\
                           0&\dots    &  & 0  &-1 & 2 \\
                      \end{array} \right).
\end{equation}




    






\end{document}
