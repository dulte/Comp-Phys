\documentclass[a4paper, 10pt]{article}
\usepackage[utf8]{inputenc}
\usepackage{verbatim}
\usepackage{listings}
\usepackage{graphicx}
\usepackage{a4wide}
\usepackage{color}
\usepackage{amsmath}
\usepackage{amssymb}
\usepackage[dvips]{epsfig}
\usepackage[T1]{fontenc}
\usepackage{cite} % [2,3,4] --> [2--4]
\usepackage{shadow}
\usepackage{hyperref}

\setcounter{tocdepth}{2}

\lstset{language=c++}
\lstset{alsolanguage=[90]Fortran}
\lstset{basicstyle=\small}
\lstset{backgroundcolor=\color{white}}
\lstset{frame=single}
\lstset{stringstyle=\ttfamily}
\lstset{keywordstyle=\color{red}\bfseries}
\lstset{commentstyle=\itshape\color{blue}}
\lstset{showspaces=false}
\lstset{showstringspaces=false}
\lstset{showtabs=false}
\lstset{breaklines}
\title{FYS3150 - Project 1}
\author{Daniel Heinesen, Halvard Sutterud, Gunnar Lange}
\begin{document}
\maketitle
\begin{abstract}

    In this project we will study speed and numerical precision of
    numerical algorithms. Given a tridiagonal matrix, we used both a
    general solving algorithm for tridiagonal matrixes and a special
    taylored 'ferrari' method for our specific problem. This is also
    compared to a cumbersome LU-decomposition of the matrix.



    Using an analytical expression for the diagonal elements in our special
    case, we were able to reduce the total of computations down to a total
    of (INSERT N) computations, where N is the number of row and column
    elements in our matrix. 

    

\end{abstract}
\tableofcontents

\addcontentsline{toc}{section}{Introduction}
\section*{Introduction }
    We will solve the one-dimensional Poisson equation with
    Dirichlet boundary conditions by reducing it to a set of differential
    equations on the form of a tridiagonal matrix. 

\section*{a)}
\subsection*{Counting the number of timesteps}
There are three different steps to compute:
\begin{equation}
\tilde{a}_i=a_i-\frac{b_{i-1}c_{i-1}}{\tilde{a}_{i-1}}
\end{equation}
\begin{equation}
\tilde{f_i}=f_i-\tilde{f}_{i-1}\frac{c_{i-1}}{\tilde{a}_{i-1}}
\end{equation}
\begin{equation}
u_i=\frac{\tilde{f_i}-b_iu_{i+1}}{\tilde{a}_i}
\end{equation}
\end{document}
