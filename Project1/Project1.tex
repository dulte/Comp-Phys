\documentclass[a4paper, 10pt]{article}
\usepackage[utf8]{inputenc}
\usepackage{verbatim}
\usepackage{listings}
\usepackage{graphicx}
\usepackage{a4wide}
\usepackage{color}
\usepackage{amsmath}
\usepackage{amssymb}
\usepackage[dvips]{epsfig}
\usepackage[T1]{fontenc}
\usepackage{cite} % [2,3,4] --> [2--4]
\usepackage{shadow}
\usepackage{hyperref}

\setcounter{tocdepth}{2}

\lstset{language=c++}
\lstset{alsolanguage=[90]Fortran}
\lstset{basicstyle=\small}
\lstset{backgroundcolor=\color{white}}
\lstset{frame=single}
\lstset{stringstyle=\ttfamily}
\lstset{keywordstyle=\color{red}\bfseries}
\lstset{commentstyle=\itshape\color{blue}}
\lstset{showspaces=false}
\lstset{showstringspaces=false}
\lstset{showtabs=false}
\lstset{breaklines}
\title{FYS3150 - Project 1}
\author{Daniel Heinesen, Halvard Sutterud, Gunnar Lange}
\begin{document}
\maketitle
\section*{a)}
\subsection*{Counting the number of timesteps}
There are three different steps to compute:
\begin{equation}
\tilde{a}_i=a_i-\frac{b_{i-1}c_{i-1}}{\tilde{a}_{i-1}}
\end{equation}
\begin{equation}
\tilde{f_i}=f_i-\tilde{f}_{i-1}\frac{c_{i-1}}{\tilde{a}_{i-1}}
\end{equation}
\begin{equation}
u_i=\frac{\tilde{f_i}-b_iu_{i+1}}{\tilde{a}_i}
\end{equation}
\end{document}