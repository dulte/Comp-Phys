\documentclass[a4paper, 10pt]{article}
\usepackage[utf8]{inputenc}
\usepackage{verbatim}
\usepackage{listings}
\usepackage{graphicx}
\usepackage{a4wide}
\usepackage{color}
\usepackage{amsmath}
\usepackage{amssymb}
\usepackage[dvips]{epsfig}
\usepackage[T1]{fontenc}
\usepackage{cite} % [2,3,4] --> [2--4]
\usepackage{shadow}
\usepackage{hyperref}

\setcounter{tocdepth}{2}

\lstset{language=c++}
\lstset{alsolanguage=[90]Fortran}
\lstset{basicstyle=\small}
\lstset{backgroundcolor=\color{white}}
\lstset{frame=single}
\lstset{stringstyle=\ttfamily}
\lstset{keywordstyle=\color{red}\bfseries}
\lstset{commentstyle=\itshape\color{blue}}
\lstset{showspaces=false}
\lstset{showstringspaces=false}
\lstset{showtabs=false}
\lstset{breaklines}
\title{FYS3150 - Project 1}
\author{Daniel Heinesen, Halvard Sutterud, Gunnar Lange}
\begin{document}
\maketitle
\begin{abstract}

    In this project we will study speed and numerical precision of
    numerical algorithms. Given a tridiagonal matrix, we used both a
    general solving algorithm for tridiagonal matrices and a special
    tailored 'ferrari' method for our specific problem. This is also
    compared to a cumbersome LU-decomposition of the matrix.



    Using an analytic expression for the diagonal elements in our special
    case, we were able to reduce the total number of FLOPS down to a total
    of (INSERT N) computations, where N is the number of row and column
    elements in our matrix. 

    

\end{abstract}
\tableofcontents

\addcontentsline{toc}{section}{Introduction}
\addcontentsline{toc}{section}{Theoretical Model}

\section*{Introduction }
    We will solve the one-dimensional Poisson equation with
    Dirichlet boundary conditions by reducing it to a set of differential
    equations on the form of a tridiagonal matrix. 

\section*{Theoretical model}
\subsection*{Discretizing the Poisson equation}
Assume that $u(x)$ is a continuous, twice differentiable function, $u(x) \in \mathbb{C}^2$. Using Taylor polynomial to the second degree, a general approximation formula to the second derivative can be derived as:
$$\frac{d^2 u}{dx^2}\approx \frac{u(x+h)+u(x-h)-2u(x)}{h^2}+O(h^2)$$
Where $h$ is a small number. Discretizing $u(x)$ at points $x_1, x_2, ..., x_n$, and  introducing the convenient notation $u(x_i)=u_i$, this formula can be rewritten in discrete form as:
$$\frac{d^2 u}{dx^2}\approx \frac{u_{i+1}+u_{i-1}-2u_{i}}{h^2}+O(h^2)$$
Where $h$ is the distance between the grid points, given by $h=1/(n+1)$. If one additionally imposes Dirichlet boundary conditions, i.e. that $u_0=u_{n+1}=0$, this formula is valid for all $x_i$, $i\in [1, n]$. Inserting this formula into the Poisson equation with right-hand side $f(x)$, and letting $f(x_i)=f_i$, gives:
$$-\frac{u_{i+1}+u_{i-1}-2u_{i}}{h^2}=f_i$$
This can be rephrased as a linear algebra problem, of the form:
$$\mathbf{A}\mathbf{u}=h^2\mathbf{f}$$
Where $\mathbf{A}$ is a $n \times n$ given by:
$$\mathbf{A}=\begin{pmatrix}
2 & -1 & 0 & \ldots &  \ldots & 0\\
-1 & 2 & -1  & 0 & \ldots & 0\\
0 & -1 & 2 &-1 & 0 & \ldots \\
0 & 0 & -1 & 2 &-1 &\ldots\\
\vdots &  & &  &\ddots & \vdots \\
0 && \ldots && -1&  2  \\
\end{pmatrix}$$
Notice that $\mathbf{A}$ is a tridiagonal matrix. This makes the general solution algorithm much simpler.
\subsection*{Solving a tridiagonal matrix problem}
A general tridiagonal matrix problem can be written as:
$$\mathbf{A}=\begin{pmatrix}
a_{1} & b_1 & 0 & \ldots &  \ldots & 0\\
c_1 & a_2 & b_2  & 0 & \ldots & 0\\
0 & c_2 & a_3 &b_3 & 0 & \ldots \\
0 & 0 & c_3 & a_4 &b_4 &\ldots\\
\vdots &  & &  &\ddots & \vdots \\
0 && \ldots && c_{n-1}&  a_n  \\
\end{pmatrix}\begin{pmatrix}
u_1\\
u_2\\
u_3\\
u_4\\
\vdots\\
u_n\\
\end{pmatrix}=\begin{pmatrix}
v_1\\
v_2\\
v_3\\
v_4\\
\vdots\\
v_n
\end{pmatrix}$$
Where, in the specific case above, $v_n=f_n/h^2$. This problem may be solved in two steps: a decomposition and forward substitution, and a backward substitution. The goal is to make each column a pivot column. For the forward substitution, it is easiest to first transform the matrix into an upper-diagonal matrix. Thus, $a_1$ needs to be zero. This can be achieved by subtracting $a_1/b_1$ times the first row from the second row. This gives:
$$\mathbf{A}=\begin{pmatrix}
a_{1} & b_1 & 0 & \ldots &  \ldots & 0\\
0 & \tilde{a_2} & b_2  & 0 & \ldots & 0\\
0 & c_2 & a_3 &b_3 & 0 & \ldots \\
0 & 0 & c_3 & a_4 &b_4 &\ldots\\
\vdots &  & &  &\ddots & \vdots \\
0 && \ldots && c_{n-1}&  a_n  \\
\end{pmatrix}\begin{pmatrix}
u_1\\
u_2\\
u_3\\
u_4\\
\vdots\\
u_n\\
\end{pmatrix}=\begin{pmatrix}
v_1\\
\tilde{v_2}\\
v_3\\
v_4\\
\vdots\\
v_n
\end{pmatrix}$$
Where:
$$\tilde{b_2}=b_2-\frac{a_1}{b_1}c_1, \quad \tilde{v_2}=v_2-\frac{a_1}{b_1}v_1$$
Notice how only $b$ change. It is therefore easy to generalize the above formulae to:
$$\tilde{b_{i+1}}=b_{i+1}$$
\subsection*{Counting the number of timesteps}
There are three different steps to compute:
\begin{equation}
\tilde{a}_i=a_i-\frac{b_{i-1}c_{i-1}}{\tilde{a}_{i-1}}
\end{equation}
\begin{equation}
\tilde{f_i}=f_i-\tilde{f}_{i-1}\frac{c_{i-1}}{\tilde{a}_{i-1}}
\end{equation}
\begin{equation}
u_i=\frac{\tilde{f_i}-b_iu_{i+1}}{\tilde{a}_i}
\end{equation}
\end{document}
