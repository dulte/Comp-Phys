\documentclass[a4paper, 10pt]{article}
\usepackage[utf8]{inputenc}
\usepackage{verbatim}
\usepackage{listings}
\usepackage{graphicx}
\usepackage{a4wide}
\usepackage{color}
\usepackage{amsmath}
\usepackage{amssymb}
\usepackage[dvips]{epsfig}
\usepackage[toc,page]{appendix}
\usepackage[T1]{fontenc}
\usepackage{cite} % [2,3,4] --> [2--4]
\usepackage{shadow}
\usepackage{hyperref}
\usepackage{titling}

\setlength{\droptitle}{-10em}   % This is your set screw

\setcounter{tocdepth}{2}

\lstset{language=c++}
\lstset{alsolanguage=[90]Fortran}
\lstset{basicstyle=\small}
\lstset{backgroundcolor=\color{white}}
\lstset{frame=single}
\lstset{stringstyle=\ttfamily}
\lstset{keywordstyle=\color{red}\bfseries}
\lstset{commentstyle=\itshape\color{blue}}
\lstset{showspaces=false}
\lstset{showstringspaces=false}
\lstset{showtabs=false}
\lstset{breaklines}
\title{FYS3150 - Project 3}
\author{Daniel Heinesen, Gunnar Lange}
\begin{document}
\maketitle
\begin{abstract}
We present numerical models of our solar system with varying levels of complexity.
\end{abstract}

\tableofcontents
\newpage
\section{Introduction}
The n-body problem is a recurring theme in many physics publications (see for example \textbf{HERE}), due to its wide applicability in many different fields, such as \textbf{REFEREENCE}.  Analytic solutions of these problems are notoriously hard to come by and are, in most situations, unattainable. 
\section{Theoretical model}\label{Theoretical_section}
\section{Methods}
\section{Results}
\section{Discussion}
\section{Conclusion and outlook}
\subsection{Conclusion}
\subsection{Outlook}


\end{document}

